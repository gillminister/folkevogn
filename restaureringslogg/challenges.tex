\chapter{Utfordringer}
\setlength{\parindent}{2ex}
Utviklingen av systemet startet for alvor i april 2016 da utviklingsteamet, med base i Trondheim, ble oppfordret til å rapportere framskritt ukentlig. Noen uker senere fant teamet ut at de ikke var helt sikre på hva de programmerte, hvordan enkeltmoduler skulle implementeres, hvilke kjøretidselementer i systemet de svarte til og til hvilke formål de ble programmert for. Med andre ord, den eksisterende dokumentasjonen på prosjektet trengte en overhaling.\par
Det kommer fram i et møte mellom utviklerne og selskapets grunnlegger Torstein den 26.05.2016 at planleggingen ikke har vært tilstrekkelig fra utviklerne sin side og det ble derfor vedtatt at kravspesifikasjonen skulle gå igjennom en oppfriskning der en liste av Use Cases med prioritering innbakt skulle skrives og gi en oversikt over kildene til kravene i dette dokumentet,  som er basert på kravene nedskrevet i den nyeste iterasjonen av kravprosessen.\par
Et stort hinder for systemets framgang har vært det faktum at to av teamets utviklere fortsatt studerer ved NTNU, og begge har følgelig måttet prioritere øvingsopplegg og eksamensperiode over utviklingsarbeidet. Sommeren 2016 er imidlertid alle tre høyst delaktige i utviklingsprosessen, ettersom de da er ferdige med alle eksamener og er fri til å implementere systemet.
