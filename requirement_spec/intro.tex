\chapter{Introduksjon}
\setlength{\parindent}{2ex}
Formålet med denne kravspesifikasjonen er å tydeliggjøre hva for slags funksjonalitet og kvaliteter programvaresystemet Agenda skal inneholde. Heretter vil vi refere til programvaresystemet Agenda som bare 'systemet'. I dette dokumentet vil leseren få innblikk i hvilke funksjonelle og ikke-funksjonelle krav systemet skal etter hvert oppfylle under implementering av det. Dokumentet beskriver også de kvalitetskrav som systemet må oppfylle, tekniske og designmessige begrensninger på dets utvikling, og eventuelle COTS-komponenter brukt til å utvikle systemet med. En tidlig testversjon av systemet skal gjøres klar til alfatesting innen medio august.\par
Systemet som utvikles er et skybasert, digitalt personalsystem som leverer vakthandteringstjenester til bedrifter innenfor næringer som elektronikk, klær, sport, dagligvare, hotell, barnehage, helse og omsorg, drosje, restaurant, cafe og uteliv. Systemets verdiforslag til bedriftene som bruker det er tidsbesparing og økt kontroll over hvem som er jobb til enhver tid. Med et digitalt peronalsystem vil man spare mye tid både på å allokere vakter til de ansatte og på manuell oppdatering av vaktlisten når ansatte vil bytte vakter seg imellom i etterkant. For de enkelte ansatte er verdiforslaget til systemet en mer effektiv måte å bytte arbeidsvakter seg imellom. Systemet åpner for en bedre kommunikasjon dem imellom for vaktbytter på en felles teknologisk plattform i nettskyen som lar seg aksesseres av en hvilken som helst moderne nettleser på både smarttelefoner og forbrukerdatamaskiner. De ansatte får en oppdatert oversikt over egne vakter opptil flere måneder frem i tid. På den måten vil hver enkelt kunne holde bedre kontroll på når de skal på jobb.\par
Fagspesifikke ord og vendinger defineres fortløpende når de nevnes. Den språklige notasjonen i de følgende tabeller er identifisert med engelske forkortelser, men beskrives utfyllende på norsk, dette fordi systemets målgruppe og dermed også våre interressenter har norsk som morsmål.